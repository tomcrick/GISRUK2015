\documentclass[11pt]{article}
\usepackage[left=25mm, right=25mm, top=25mm, bottom=25mm, includehead=true, includefoot=true]{geometry}

\usepackage{graphicx}
\usepackage{url}
\usepackage{natbib} % For referencing
\usepackage{authblk} % For author lists
\usepackage[parfill]{parskip} % Line between paragraphs

\pagenumbering{gobble} % Turn off page numbers

% Make all headings the same size (11pt):
\usepackage{sectsty}
\sectionfont{\normalsize}
\subsectionfont{\normalsize}
\subsubsectionfont{\normalsize}
\paragraphfont{\normalsize}

\renewcommand{\abstractname}{Summary} % Make 'abstract' be called 'Summary'

% This makes links and bookmarks in the pdf output (should be last usepackage command because it overrides lots of other commands)
\usepackage[pdftex]{hyperref} 
\hypersetup{pdfborder={0 0 0} } % This turns off the stupid colourful border around links


\title{Data Exploration with GIS Viewsheds and Social Network Analysis}

\author[1]{Giles Oatley\thanks{goatley@cardiffmet.ac.uk}}
\author[1]{Tom Crick\thanks{tcrick@cardiffmet.ac.uk}}
\author[2]{Ray Howell\thanks{ray.howell@southwales.ac.uk}}
\affil[1]{Department of Computing, Cardiff Metropolitan University, UK}
\affil[2]{Faculty of Business and Society, University of South Wales, UK}

\date{ }

\renewcommand\Authands{ and } % correct last comma in author list

\begin{document}

\maketitle

\begin{abstract}
\centering

We present a novel exploratory method combining line of sight
visibility (viewshed analysis) and techniques from social network
analysis to investigate archaeological data. At increasing distances
different nodes are connected creating a set of networks, which are
subsequently described using centrality measures and clustering
coefficients. Networks with significant properties are examined in
more detail. We use this method to investigate the placement of
hillforts (nodes) in the Gwent region of south-east Wales, UK. We are
able to determine distances that support significant transitions in
network structure that could have significant archaeological validity.

$ $ \\ {\bf KEYWORDS:} Geographic networks, archaeological nodes,
viewshed analysis, data mining, social network analysis

\end{abstract}


\section*{Extended Abstract}

We present a novel exploratory method that combines line of sight
visibility (viewshed analysis) with techniques from social network
analysis to investigate archaeological data. Within data mining exist
the fields of graph-based and spatial-based data mining. Graph-based
data mining~\citep{cook+holder:2006} has a close cousin in the long
established field of social network analysis, a set of metrics that
operates over graphs (networks) created from
links~\citep{wasserman+faust:1995}. Metrics include those to find
clusters within networks, to find points that have significant
properties, for instance how central a point is. Spatial data mining
likewise has an extensive history~\citep{lu-et-al:1993}, and is the
discovery of interesting patterns from spatial datasets.

At increasing distances different nodes are connected creating a set
of networks, which are subsequently described using centrality
measures and clustering coefficients. Networks with significant
properties are examined in more detail. We use this method to
investigate the placement of hillforts (nodes) in the Gwent region of
south-east Wales, UK. Our methodology is applied to the area of the
Iron Age tribe known as the Silures, described as a `resilient and
sophisticated clan based tribal
confederation'~\citep{howell:2009}. Our preliminary investigation
focuses on the Gwent region with a study area which roughly
approximates the county as constituted between 1974 and 1996.
Figure~\ref{fig:hillforts} shows the placement of 30 hillforts in this
region. We are able to determine distances that support significant
transitions in net-work structure that could have archaeological
validity.  Our study uses both geographical and graph/network
structures, and presents an exploratory methodology within which to
discover significant distances underlying network creation. While
based on archaeological informatics, the approach has a more general
use, for instance neural architectures, transportation networks, and
other forms of geographical networks.

This research lies in the intersection of spatial and graph-based
data. Related work includes that of the physics literature on
geographical networks~\citep{ben-avraham-et-al:2003}, architectural
analysis and the isovist literature including visibility
graphs~\citep{steadman:1973,llobera:1996,turner-et-al:2001}, and the
authors' recent work incorporating kernel density estimation into the
betweenness social network
metric~\citep{oatley+crick_asonam2014,oatley+crick_fosintsi2014}. The
data used includes the Iron Age hillfort data, provided from the
Historic Environment Records\footnote{Archwilio, the Historical
Environment Records of the Welsh Archaeological Trusts:
\url{http://www.archwilio.org.uk/}}, and a Digital Elevation Model
based on the Shuttle Radar Topography Mission data (UK SRTM
DEM)\footnote{UK SRTM DEM created by Addy Pope. Spatial Reference
System--Great Britain National Grid:
\url{http://edina.ac.uk/projects/sharegeo/}} with 90m horizontal
resolution.

We develop connectivity between Iron Age hillforts based on viewsheds
and an increasing distance threshold. A viewshed is the area of land
that is within line of sight from a fixed viewing position. We analyse
the generated set of networks of connected hillforts using social
network analysis, and use the metrics to inform theories of possible
use and communication between hillforts.  Degree centrality is
simplest and is a count of the number of links to other nodes in the
network. Closeness however is a measure of how close a node is to all
other nodes in a network~\citep{sabidussi:1966}. It is the mean of the
shortest paths between a node and all other nodes reachable from
it. Betweenness is the extent to which a node lies between other nodes
in the network and is equal to the number of shortest paths from all
nodes to all others that pass through that
node~\citep{freeman:1977}. This measure takes into account the
connectivity of the node's neighbours, giving a higher value for nodes
which bridge clusters.

\begin{figure}[!htp]
\centering
\includegraphics[width=0.8\textwidth]{images/hillforts.png}
\caption{Hillforts in south-east Wales. Hillforts are displayed as
  white crosses on the front contour display. The same terrain and
  hillforts (white circles) are displayed behind on a Digital
  Elevation Model (DEM). The DEM display shows that there are many
  other sites that could have been used for placement of hillforts.}
\label{fig:hillforts}
\end{figure}

We explore using a local clustering
coefficient~\citep{watts+strogatz:1998} quantifying how close a
networks nodes’ neighbours are to being a clique (fully connected).
Viewsheds are generated for each hillfort, in order to determine
intervisibility between every hillfort. We are then able to determine
which hillforts are intervisible at any given distance threshold.  In
this way we investigate networks of hillforts at different distance
values examining the clustering coefficient and betweenness measures.

\begin{figure}[!hp]
\centering
\includegraphics[width=0.7\textwidth]{images/interestingnetworks.png}
\caption{Interesting networks. 1: 5km. 2: 10km. 3: 15km. 4: 20km.}
\label{fig:intnetworks}
\end{figure}

This reveals several interesting transition points (see
Figure~\ref{fig:intnetworks}) in connectivity, including localised
clusters being evident, connectivity between larger regions, and
connectivity along key geographic features such as along a shoreline
and up waterways.  In previous studies `significant' distances and
decay values have been determined a priori. We, however, examine the
centrality of individual nodes (hillforts) in these networks with the
most significant values. We are interested in discovering interesting
patterns and clusters and then investigating them {\emph{a
posteriori}} for (archaeological) validity.  Among preliminary
conclusions arising from this first phase of investigation is that the
methodology employed can effectively inform our understanding of Iron
Age social structures. For example, viewshed analysis confirms
hypothesised clan-based clustering of hillforts in the region with
extensive line of sight communication, not only within clusters, but
also with other hillfort groupings.  The model of a clan-based
confederation with regional emphasis, and possibly variation, but with
wider connectivity sufficient to allow the cohesion necessary to have
resisted the Roman advance so effectively seems wholly appropriate.
Future work will utilise fuzzy viewsheds instead of the standard
binary viewshed, with distance decay functions based on the limits of
normal human vision and such features as the size of people,
livestock, distances that smoke plumes can be seen and so on. We will
also consider the integration of least-cost paths in landscapes.


\section*{Biography}

{\textbf{Dr Giles Oatley}} is a Reader in Intelligent Systems at
Cardiff Metropolitan University. He has developed decision support
systems based on behavioural models from data mining, primarily for UK
police forces, supported by the EPSRC, Home Office, HEFCE, Nuffield
Foundation, and DTI. He has a broad interest in anthropology and
psychology, especially mindfulness and psychoanalysis.

{\textbf{Dr Tom Crick}} is a Senior Lecturer in Computing Science at
Cardiff Metropolitan University. His research is naturally
interdisciplinary: optimisation, intelligent systems, data science and
analytics, high performance computing and reproducibility. He is the
Nesta Data Science Fellow, a 2014 Fellow of the Software
Sustainability Institute and a member of {\emph{HiPEAC}}, the European
FP7 Network of Excellence on High Performance and Embedded
Architecture and Compilation.

{\textbf{Professor Ray Howell}} is Professor of Welsh Antiquity and
Director of the South Wales Centre for Historical and
Interdisciplinary Research at the University of South Wales. He is a
Fellow of the Society of Antiquaries of London. He is also Chairman of
the Glamorgan Gwent Archaeological Trust and the Glamorgan Gwent
Historic Environment Record Charitable Trust.



\bibliographystyle{apa}
\bibliography{gisruk2015}

\end{document}
